% -------------------------------------------------------
% Daten für die Arbeit
% Wenn hier alles korrekt eingetragen wurde, wird das Titelblatt
% automatisch generiert. D.h. die Datei titelblatt.tex muss nicht mehr
% angepasst werden.

% Titel der Arbeit auf Deutsch
\newcommand{\hsmatitelde}{Konzipierung und Entwicklung einer Progressive Web App zum Herunterladen, Verwalten und Abspielen von Audio-Medien zur Offlinenutzung mit Angular}

% Titel der Arbeit auf Englisch
\newcommand{\hsmatitelen}{Design and Developement of a progressive web app to download, manage and play audio files for offline use with Angular}

% Weitere Informationen zur Arbeit
\newcommand{\hsmaort}{Mannheim}    % Ort
\newcommand{\hsmaautorvname}{Martin} % Vorname(n)
\newcommand{\hsmaautornname}{Schalter} % Nachname(n)
\newcommand{\hsmadatum}{19.08.2020} % Datum der Abgabe
\newcommand{\hsmajahr}{2020} % Jahr der Abgabe
\newcommand{\hsmafirma}{biblepool gUG, Trossingen} % Firma bei der die Arbeit durchgeführt wurde
\newcommand{\hsmabetreuer}{Prof. Dr. Thomas Specht, Hochschule Mannheim} % Betreuer an der Hochschule
\newcommand{\hsmazweitkorrektor}{Christian Perian, biblepool gUG} % Betreuer im Unternehmen oder Zweitkorrektor
\newcommand{\hsmafakultaet}{I} % I für Informatik oder E, S, B, D, M, N, W, V
\newcommand{\hsmastudiengang}{IB} % IB IMB UIB CSB IM MTB (weitere siehe titleblatt.tex)

% Zustimmung zur Veröffentlichung
\setboolean{hsmapublizieren}{true}   % Einer Veröffentlichung wird zugestimmt
\setboolean{hsmasperrvermerk}{false} % Die Arbeit hat keinen Sperrvermerk

% -------------------------------------------------------
% Abstract
% Achtung: Wenn Sie im Abstrakt Anführungszeichen verwenden wollen, dann benutzen Sie
%          nicht "` und "', sondern \enquote{}. "` und "' werden nicht richtig
%          erkannt.

% Kurze (maximal halbseitige) Beschreibung, worum es in der Arbeit geht auf Deutsch
\newcommand{\hsmaabstractde}{
Die vorliegende Arbeit untersucht die Möglichkeiten der Offline-Funktionalitäten von Progressive Web Apps. Dazu wird für das Webportal CROSSLOAD ein Konzept zur Offline-Nutzung von Audio-Inhalten entworfen und anschließend prototypisch die Machbarkeit gezeigt. Zum Finden der besten Architektur wurden viele in Frage kommenden Funktionen von modernen Browsern erklärt und anhand von zuvor erarbeiteten Anforderungen verglichen. Dadurch konnten fast alle Anforderungen ohne Einschränkung umgesetzt werden. Die APIs der modernen Webbrowser bieten ausreichend Möglichkeiten eine Offline-Funktionalität gut umzusetzen. In manchen Bereichen sind native Apps den Progressive Web Apps noch voraus, jedoch kommen immer mehr Funktionen in den Browsern hinzu, die diese Nachteile ausgleichen.
}

% Kurze (maximal halbseitige) Beschreibung, worum es in der Arbeit geht auf Englisch
\newcommand{\hsmaabstracten}{
This thesis examines the possibilities of offline functionalitites of Progressive Web Apps. Therefore the web portal CROSSLOAD was extended with an offline functionality. A concept for saving and managing of offline content was designed and prototypically implemented. In order to find the best architecture, many functions of modern browsers were explained and compared on the basis of previously developed requirements. As a result, almost all requirements could be implmented without restriction. The APIs of modern web browsers offer sufficient possibilities to implement offline functionality well. In some areas native apps are still ahead of progressive web apps, but more and more features are being added to modern browsers to compensate these disadvantages.
}
