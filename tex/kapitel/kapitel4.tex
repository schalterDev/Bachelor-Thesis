\chapter{Konzeption}
\label{Kap4}

\section{Lokale Datenspeicherung}
Moderne Webbrowser stellen viele verschiedene Lösungen zum lokalen Speichern von Daten zur Verfügung. In diesem Kapitel werden diese vorgestellt und miteinander verglichen. Zuletzt wird eine Entscheidung darüber getroffen, welche Technologien für CROSSLOAD am geeignetsten sind und deshalb verwendet werden.

Vergleichskriterien zur lokalen Datenspeicherung ergeben sich aus den Anforderungen. Hinter jedem Kriterium steht in Klammern die Nummer der Anforderung aus \autoref{Kap3:Anforderungen}:
\begin{enumerate}
	\item Ist die \ac{API} asynchron oder nur synchron nutzbar? Eine nur synchron nutzbare Lösung könnte zum Einfrieren der Nutzeroberfläche führen (Anforderung 14)
	\item Maximal zulässige Speichermenge (Anforderung 11, 12, und 13)
	\item Mögliche Datentypen. Können Bilder, Audiodateien und Text abgespeichert werden? (Anforderung 9 und 11)
	\item Wie gut wird das Durchsuchen, Filtern und Löschen von Inhalten unterstützt? (Anforderung 5, 6, 7 und 8)
	\item Welche Browser unterstützten diese Funktionalitäten?
\end{enumerate}

\subsection{Web Storage}
Die Web Storage \ac{API} bietet die Möglichkeit, Schlüssel-Wert-Paare im Browser zu speichern \autocite{mdn-web-storage}. Werte werden immer mit einem Schlüssel gespeichert oder über einen Schlüssel geladen \autocite{mdn-web-storage}.

Alle relevanten Browser implementieren diese \ac{API} \autocite{mdn-web-storage}, es gibt aber auch einige Einschränkungen:
\begin{itemize}
	\item Zugriffe sind nur synchron möglich \autocite{Hajian2019}. Das heißt: Wenn große Datenmengen gespeichert oder geladen werden, kann das Browserfenster für eine Zeit lang einfrieren
	\item Als Schlüssel und Werte können jeweils nur Strings gespeichert werden \autocite{Hajian2019}. Sollen JavaScript-Objekte gespeichert werden müssen diese in JSON umgewandelt werden.
	\item Es können maximal 5 \ac{MB} an Daten gespeichert werden \autocite{mdn-web-storage}
\end{itemize}

\bild{architekturen/web-storage}{14cm}{Aufbau der Web Storage \ac{API}}

\autoref{architekturen/web-storage} zeigt die Web Storage \ac{API} mit SessionStorage und LocalStorage. Diese beiden Speicher unterscheiden sich nur darin, wie lange Daten gespeichert werden \autocite{Hajian2019} \autocite{mdn-web-storage}. Die Daten vom SessionStorage werden gelöscht, sobald die Sitzung auf der Website vorbei ist, das heißt der Browser oder der Tab geschlossen wird \autocite{Hajian2019} \autocite{mdn-web-storage}. Der LocalStorage bleibt über mehrere Sitzungen erhalten und wird nicht automatisch gelöscht \autocite{Hajian2019} \autocite{mdn-web-storage}. In beiden Speichern können Schlüssel-Wert-Paare als Text gespeichert werden \autocite{Hajian2019} \autocite{mdn-web-storage}.

\subsection{Filesystem API}
Die Filesystem \ac{API} und FileWriter \ac{API} bieten dem Browser die Möglichkeit, Dateien in ein virtuelles Dateisystem abzulegen und von dort wieder zu laden \autocite{Hajian2019} \autocite{storage-for-the-web}. Das \ac{MDN} nennt diese \ac{API} File and Directory Entries \ac{API} \autocite{mdn-file-system}. 

\bild{architekturen/file-system-api}{14cm}{Struktur der Filesystem \ac{API}}

In \autoref{architekturen/file-system-api} ist die Struktur der Filesystem \ac{API} zu sehen. Es gibt Ordner und Dateien. Ein Ordner kann Dateien und Unterordner enthalten. Die \ac{API} erlaubt eine Navigation in den Ordnern, wie zum Beispiel: gehe in den übergeordneten Ordner \autocite{mdn-file-system}. Außerdem lassen sich alle Dateien und Unterordner eines Ordners auflisten \autocite{mdn-file-system}. Dateien können geladen und gespeichert werden \autocite{mdn-file-system}. Diese Dateien und Ordner liegen nur in einem virtuellen Dateisystem und sind in dieser Ordnerstruktur nicht auf der Festplatte des Benutzers wiederzufinden \autocite{mdn-file-system}.

Der Vorteil dieser \ac{API} ist der Umgang mit \acp{blob}, wie zum Beispiel Audiodateien. Diese können leicht gespeichert, geladen und manipuliert werden \autocite{mdn-file-system}. Es können ebenso andere Datentypen wie Strings gespeichert werden \autocite{mdn-file-system}. Außerdem gibt es eine \ac{API} für synchrone und asynchrone Zugriffe \autocite{Hajian2019}. 

Diese \ac{API} wurde noch nicht standardisiert \autocite{mdn-file-system} \autocite{caniuse-filesystem}, deswegen unterstützen noch nicht sehr viele Browser diese \ac{API}. Bisher ist in Chrome und in allen Chromium basierten Browser diese Funktion verfügbar \autocite{caniuse-filesystem}.

\subsection{Web SQL}
Web SQL ist eine \ac{API}, die es erlaubt Daten in einer Datenbank zu speichern und diese Daten mit einer SQL ähnlichen Sprache zu durchsuchen \autocite{w3-web-sql}. Alle Anfragen sind asynchron \autocite{Hajian2019}.

Diese \ac{API} wurde nie von allen Browsern implementiert und ist mittlerweile deprecated, soll also nicht mehr verwendet werden \autocite{Hajian2019}.

Aufgrund der Tatsache, dass Web SQL deprecated ist, wird diese Technologie auch keinen Einsatz bei CROSSLOAD finden und wird hier nicht weiter erläutert. 

\subsection{IndexedDB}
IndexedDB ist eine key-value NoSQL objekt-orientierte Datenbank zum Speichern von großen und vielen Dateien \autocite{Hajian2019}. Konkret heißt das, dass Objekte mit einem Schlüssel in der Datenbank abgelegt werden können und diese über den angegebenen Schlüssel wieder auffindbar sind. Dabei werden viele unterschiedliche Datentypen (boolean, number, string, date, object, array, regexp, undefined, null, \ac{blob}) \autocite{mdn-indexeddb}, Transaktionen und Indexe zum schnelleren Durchsuchen und Filtern unterstützt \autocite{Sheppard2017}. 

\autoref{architekturen/indexedDB} zeigt den Aufbau von IndexedDB. Es können mehrere Datenbanken erstellt werden und jede Datenbank hat mehrere object stores \autocite{mdn-indexeddb}. Ein object store ist mit einer Tabelle in einer relationalen Datenbank vergleichbar. Transaktionen sind innerhalb einer Datenbank über mehrere object stores möglich \autocite{mdn-indexeddb}. Jeder object store kann Schlüssel-Wert-Paare speichern \autocite{Sheppard2017}. Als Werte können JavaScript-Objekte, Zahlen, Texte, Bilder, Audiodateien und vieles mehr gespeichert werden \autocite{mdn-indexeddb}. Wenn Werte JavaScript-Objekte sind, kann auf ein Attribut ein Index gelegt werden \autocite{mdn-indexeddb}. In diesem Beispiel könnte das Attribut \textit{id} oder \textit{titel} oder auch beide indexiert werden.

\bild{architekturen/indexedDB}{14cm}{Architektur der IndexedDB}

IndexedDB kann asynchron verwendet werden \autocite{Hajian2019} \autocite{mdn-indexeddb} und alle relevanten Browser unterstützen die \ac{API} \autocite{mdn-indexeddb-api}. Die maximale Speicherkapazität der Datenbank hängt vom Browser ab und wird in Kapitel~\ref{Kap4:Datenmenge} näher behandelt.

Anfragen an die Datenbank können direkt eine Sortierung oder Suchkriterien enthalten \autocite{mdn-indexeddb}. Dabei ist zu beachten, dass die Textsuche limitiert ist und keine Anfragen zum Finden eines einzelnen Wortes in einem ganzen Text unterstützt \autocite{mdn-indexeddb}. 

Das Programmieren mit der \ac{API} von IndexedDB ist komplex, weil mit Events gearbeitet wird und nicht wie in modernen \acp{API} mit Promises \autocite{Hajian2019}. Deswegen existieren viele Bibliotheken, die eine Verwendung der \ac{API} erleichtern, wie zum Beispiel LocalForage oder Dexie.js \autocite{Hajian2019} \autocite{mdn-indexeddb}.

\subsection{Cache API}
Die Cache \ac{API} bietet die Möglichkeit, Netzwerkanfragen zwischenzuspeichern \autocite{mdn-cache-api}. Über Request-Objekte können Response-Objekte gespeichert und geladen werden \autocite{mdn-cache-api}.

\bild{architekturen/cache-api}{14cm}{Aufbau der Cache \ac{API}}

\autoref{architekturen/cache-api} zeigt den Aufbau der Cache \ac{API}: Es können mehrere Caches angelegt werden \autocite{mdn-cache-api}. In einem Cache können mehrere Request-Response-Paare gespeichert werden \autocite{mdn-cache-api}. Ein Request-Objekt enthält die URL, die Methode (GET, POST, PUT, ...), einen body und optional noch mehr Parameter \autocite{mdn-request}. Das Response-Objekt ist ähnlich aufgebaut, enthält aber zum Beispiel auch einen Statuscode \autocite{mdn-response}.

Die \ac{API} ist in allen relevanten Browsern verfügbar \autocite{mdn-cache-api}. Oft wird die Cache \ac{API} in Verbindung mit Service Workern verwendet. Die maximale Speicherkapazität des Caches hängt vom Browser ab und wird in Kapitel~\ref{Kap4:Datenmenge} näher erläutert. 

\subsection{Maximale Datenmenge}
\label{Kap4:Datenmenge}
Es gibt keine einheitliche Regelung, wie viel Speicher von einer Webseite genutzt werden darf, deswegen hat jeder Browser seine eigenen Regeln \autocite{storage-for-the-web}. 

Chrome und Firefox verfolgen die gleiche Strategie beim Vergeben von Speicherplatz: Es gibt einen shared pool, den sich alle Websites teilen \autocite{storage-for-the-web} \autocite{mdn-browser-storage-limit}. Also eine maximale Datenmenge, die der Browser generell speichert. Außerdem gibt es ein group limit, wobei jede \ac{TLD} eine eigene Gruppe darstellt \autocite{mdn-browser-storage-limit}. Zu einer Gruppe würden also hs-mannheim.de, www.hs-mannheim.de und bib.hs-mannheim.de gehören. Eine weiter Gruppe wäre zum Beispiel uni-mannheim.de Das group limit hängt unter anderem vom shared pool ab. Die genauen Limits sind in \autoref{Kap4:Speicherlimits} angegeben.

Safari erlaubt eine Nutzung von bis zu 1 \ac{GB}, wenn dieses Limit erreicht ist, wird der Nutzer gefragt, ob er mehr Speicher erlauben möchte \autocite{storage-for-the-web}. 

\begin{table}
  \renewcommand{\arraystretch}{1.2}
  \centering
  \sffamily
  \begin{footnotesize}
    \begin{tabular}{l l l}
      \toprule
      \textbf{Browser} & \textbf{shared pool} & \textbf{group limit} \\
      \midrule
      \emph{Chrome} & bis zu 60\% der Festplattengröße & 100\% des shared pools \\
      \emph{Firefox} & bis zu 50\% der Festplattengröße & 20\% vom shared pool aber maximal 2 \ac{GB} \\
      \emph{Safari} & nicht bekannt & bis zu 1 \ac{GB}, durch Nutzerbestätigung erweiterbar \\
      \bottomrule
    \end{tabular}
  \end{footnotesize}
  \rmfamily
  \caption{Speicherlimits der Browser}
  \label{Kap4:Speicherlimits}
\end{table}

Die StorageManager \ac{API} kann verwendet werden, um den verfügbaren sowie den bisher genutzten Speicherplatz abzufragen \autocite{storage-for-the-web}. Diese Angaben müssen nicht genau sein, sondern bieten lediglich einen Richtwert \autocite{storage-for-the-web}. Diese Funktion wird allerdings nicht von allen Browsern unterstützt. Im Moment wird die \ac{API} von Firefox und Chrome, nicht aber von Safari unterstützt \autocite{mdn-storage-api}.

\subsection{Mögliche Architekturen}
In Tabelle~\ref{Kap4:Datenspeicherung} werden die vorgestellten Möglichkeiten zum Speichern von Daten nach bereits genannten Kriterien verglichen: maximale Datenmenge, mögliche Datentypen, synchron / asynchron, Browsersupport, Aufwand für Suchen und Filtern. Bei fehlenden Feldern in Web SQL wurde nicht weiter recherchiert, weil diese Technologie veraltet ist und nicht mehr verwendet werden soll.

\begin{sidewaystable}[h]
  \renewcommand{\arraystretch}{1.2}
  \centering
  \sffamily
  \begin{footnotesize}
    \begin{tabularx}{1.0\textwidth}{l L L L l l L}
      \toprule
      \textbf{Technologie} & \textbf{Datenmenge} & \textbf{Datentypen} & \textbf{(a)synchron} & \textbf{Browsersupport} & \textbf{Unterstützung für Suchen und Filtern} & \textbf{Anmerkung} \\
      \midrule
      \emph{Web Storage \ac{API}} & max. 5 \ac{MB} & String & nur synchron & alle & nicht vorhanden & \\
      \emph{Filesystem \ac{API}} & max. Speicherkapazität des Browsers & \acp{blob} und weitere & synchron und asynchron & nur Chrome & nicht vorhanden &  \\
      \emph{Web SQL} & - & . & synchron und asynchron & teilweise & vorhanden & veraltet \\
      \emph{IndexedDB} & max. Speicherkapazität des Browsers & einfache Datentypen, Objekte, \acp{blob} & asynchron & alle & vorhanden & \\
      \emph{Cache \ac{API}} & max. Speicherkapazität des Browsers & Request- und Response-Objekte & asynchron & alle & nicht vorhanden & \\
      \bottomrule
    \end{tabularx}
  \end{footnotesize}
  \rmfamily
  \caption{Vergleich der APIs zur lokalen Datenspeicherung}
  \label{Kap4:Datenspeicherung}
\end{sidewaystable}

\clearpage

Web SQL ist deprecated und soll deswegen in CROSSLOAD keine Anwendung finden. Außerdem ist das Speichern von großen Datenmengen dringend erforderlich, weshalb jede Architektur mindestens eine der folgenden Technologien benutzen muss: Filesystem \ac{API}, IndexedDB, Cache \ac{API}.

Es werden nun mögliche Architekturen vorgestellt und miteinander verglichen. Als Kriterien gelten die selben wie bisher: maximale Datenmenge, mögliche Datentypen, synchron / asynchron, Browsersupport, Aufwand für Suchen und Filtern.

\subsubsection{Cache API und Web Storage}
Die erste Architektur ist in \autoref{architekturen/architektur-cache-api-web-storage} zu sehen. Es wird die Cache \ac{API} zum Speichern der Audiodateien und der Web Storage zum Speichern der Metadaten verwendet. Metadaten sind in diesem Fall alle Information, die zu einem Inhalt gehören, abgesehen von der Audiodatei. Der Web Storage speichert alle Ids der bisher favorisierten Inhalte in einem Array. Außerdem hat jede Id einen eigenen Eintrag im Web Storage, der die Metadaten enthält. Dieser Ansatz hat den Vorteil, dass schnell überprüft werden kann, ob ein einzelner Inhalt favorisiert ist oder nicht. Das Abrufen aller favorisierten Inhalte dauert hingegen lange, weil viele verschiedene Einträge aus dem Web Storage geladen werden müssen. Als alternativen Ansatz könnte man den in \autoref{metadaten-array} beschriebenen umsetzten. Hier werden alle Metadaten in einem einzigen Array gespeichert. Das Laden aller Inhalte ist jetzt schneller. Das Speichern oder Laden eines einzelnen Inhalts dauert dagegen länger, weil immer das gesamte Array geladen oder gespeichert werden muss.

\bild{architekturen/architektur-cache-api-web-storage}{14cm}{Architektur mit Cache API und Web Storage}

\begin{lstlisting}[language=JavaScript,caption={Speichern der Metadaten in einem Array},label={metadaten-array}]
    favoriten: [
    	{
    		id: "id1",
    		titel: "Titel des Inhalts",
    		erstelltAm: "05.06.2020",
    		// weitere Metadaten
    	},
    	{
    		id: "id2",
    		// weitere Metadaten
    	}
    ]
\end{lstlisting}

In \autoref{diagramme/ablauf-cache-api-locale-storage-favorisieren} ist der Ablauf dargestellt, wenn ein Nutzer einen Inhalt favorisiert. Sobald der Nutzer einen Inhalt favorisiert, werden die Metadaten abgerufen und direkt abgespeichert. Anschließend beginnt der Download der Audiodatei, welche in der Cache \ac{API} gespeichert wird. Sobald das Herunterladen und Speichern erfolgreich war, wird dies in den Metadaten aktualisiert und der Nutzer wird benachrichtigt.

\bild{diagramme/ablauf-cache-api-locale-storage-favorisieren}{14cm}{Ablauf für das Favorisieren mit Cache API und Web Storage}

In \autoref{diagramme/ablauf-cache-api-locale-storage-abspielen} wird der Ablauf für das Abspielen eines Inhaltes gezeigt. Damit der Nutzer auf den abzuspielenden Inhalt kommt, lässt er sich alle favorisierten Inhalte anzeigen. Die Metadaten aller Inhalte müssen dementsprechend aus dem Web Storage geladen werden und anschließend angezeigt werden. Der Nutzer wählt einen dieser Inhalte aus. Die Audiodatei wird nun aus der Cache \ac{API} geladen und abgespielt.

\bild{diagramme/ablauf-cache-api-locale-storage-abspielen}{14cm}{Ablauf für das Abspielen mit Cache API und Web Storage}

Der Vorteil dieser Architektur liegt in der Einfachheit der benutzten \acp{API}. Sowohl die WebStorage \ac{API}, sowie die Cache \ac{API} sind sehr einfach zu verstehen und zu verwenden. Außerdem sind beide Technologien in allen relevanten Browsern verfügbar. Die Nachteile dieser Architektur sind:

\begin{itemize}
\item Der Zugriff auf den Web Storage ist synchron. Bei vielen Daten kann dies zu längeren Ladezeiten kommen und währenddessen ist das Portal nicht bedienbar. 
\item Durch die Verwendung des Web Storages werden entweder Aktualisierungen der Metadaten oder das Abrufen aller Metadaten ineffizient. 
\item Das Speicherlimit des Web Storages beträgt 5 \ac{MB}. Bei Metadaten, die bis zu 50 \ac{kB} groß sind, können im schlechtesten Fall 100 Einträge gespeichert werden. 
\item Wenn Inhalte durchsucht oder gefiltert werden müssen, muss dies manuell nach dem Laden aller Metadaten erfolgen.
\end{itemize}

\subsubsection{Cache API und IndexedDB}
\bild{architekturen/architektur-cache-api-indexed-db}{14cm}{Architektur mit Cache API und IndexedDB}

\autoref{architekturen/architektur-cache-api-indexed-db} zeigt die Architektur mit Cache \ac{API} und IndexedDB. Die Cache \ac{API} wird zum Speichern der Audiodateien verwendet und in der IndexedDB werden alle Metadaten gespeichert.

Der Ablauf zum Favorisieren und Abspielen eines Inhaltes bleibt wie in \autoref{diagramme/ablauf-cache-api-locale-storage-favorisieren} und \autoref{diagramme/ablauf-cache-api-locale-storage-abspielen} beschrieben mit dem kleinen Unterschied, dass der Web Storage durch IndexedDB ersetzt wird. 

Durch die Verwendung von IndexedDB ist das Abrufen und Speichern der Metadaten nicht synchron, sondern asynchron. Außerdem bietet IndexedDB die Möglichkeit, Inhalte zu sortieren und zu filtern. IndexedDB ist in allen relevanten Browsern verfügbar .

\subsubsection{Filesystem API und IndexedDB}
\bild{architekturen/architektur-file-system-api-indexed-db}{14cm}{Architektur mit Filesystem API und IndexedDB}

In dieser Architektur, welche in \autoref{architekturen/architektur-file-system-api-indexed-db} beschrieben ist, werden Audiodateien mit der Filesystem \ac{API} abgespeichert. Metadaten werden in der IndexedDB gespeichert, wo sie auch sortiert und gefiltert werden können. In dem virtuellen Dateisystem wird ein Ordner \textit{favoriten} angelegt. In diesem Ordner gibt es Unterordner für jeden Inhalt, welche die heruntergeladene Audiodatei enthalten. 

Der Nachteil dieser Architektur ist die Browserkompatibilität. Die Filesystem \ac{API} wird bisher nur in Chrome unterstützt.

\subsubsection{IndexedDB}
\bild{architekturen/architektur-indexed-db}{14cm}{Architektur mit IndexedDB}

\autoref{architekturen/architektur-indexed-db} zeigt die Architektur zum Speichern der Metadaten und Audiodaten mit IndexedDB. Sowohl Metadaten als auch die Audiodatei werden in der Datenbank abgespeichert. Es werden zwei object stores verwendet. Ein object store speichert alle Metadaten, während der andere alle Audiodateien speichert. Es wäre auch möglich, die Audiodateien mit im object store \textit{favoriten} zu speichern. Dies wäre jedoch schlecht für die Geschwindigkeit der Seite, wenn nur die Metadaten ohne die Audiodatei benötigt werden.

\autoref{diagramme/ablauf-indexed-db-abspielen} zeigt dies am Beispiel für das Abspielen eines Inhalts auf CROSSLOAD. Wenn der Nutzer sich alle Inhalte anzeigen lässt, werden zuerst nur die Metadaten benötigt. Erst wenn ein Inhalt ausgewählt wurde, wird die Audiodatei zu einem Inhalt benötigt. 

\bild{diagramme/ablauf-indexed-db-abspielen}{14cm}{Ablauf für das Abspielen mit IndexedDB}

In \autoref{diagramme/ablauf-indexed-db-favorisieren} ist der Ablauf zum Favorisieren eines Inhalts gezeigt. Sowohl die Metadaten als auch die Audiodatei wird in der IndexedDB gespeichert.

\bild{diagramme/ablauf-indexed-db-favorisieren}{14cm}{Ablauf für das Favorisieren mit IndexedDB}

Das Abrufen und Speichern aller Daten geschieht nicht synchron, sondern asynchron. Außerdem bietet IndexedDB die Möglichkeit, Metadaten sortiert oder gefiltert abzurufen.

\begin{sidewaystable}[h]
  \renewcommand{\arraystretch}{1.2}
  \centering
  \sffamily
  \begin{footnotesize}
    \begin{tabularx}{1.0\textwidth}{l L L L l l}
      \toprule
      \textbf{Architektur} & \textbf{Datenmenge} & \textbf{Datentypen} & \textbf{(a)synchron} & \textbf{Browsersupport} & \textbf{Unterstützung für Suchen und Filtern} \\
      \midrule
      \emph{Cache \ac{API} und Web Storage} & max. 5 \ac{MB} Metadaten & Metadaten müssen in einen String umgewandelt werden & Metadaten nur synchron & alle & nicht vorhanden \\
      \emph{Cache \ac{API} und IndexedDB} & max. Speicherkapazität des Browsers & \acp{blob} und weitere & synchron und asynchron & alle & vorhanden \\
      \emph{Filesystem \ac{API} und IndexedDB} & max. Speicherkapazität des Browsers & \acp{blob} und weitere & synchron und asynchron & teilweise & vorhanden \\
      \emph{IndexedDB} & max. Speicherkapazität des Browsers & \acp{blob} und weitere & asynchron & alle & vorhanden \\
      \bottomrule
    \end{tabularx}
  \end{footnotesize}
  \rmfamily
  \caption{Vergleich der Architekturen zur lokalen Datenspeicherung}
  \label{Kap4:Architekturen-Vergleich}
\end{sidewaystable}

\clearpage

\subsubsection{Entscheidung}
In Tabelle~\ref{Kap4:Architekturen-Vergleich} werden alle vorgestellten Architekturen miteinander verglichen. Die Architektur \textit{Cache API und IndexedDB} sowie \textit{IndexedDB} erfüllen alle Kriterien: Es wird die maximal zulässige Speicherkapazität des Browsers ausgenutzt, alle relevanten Datentypen können asynchron geladen und gespeichert werden, alle relevanten Browser unterstützten die \acp{API} und das Filtern, beziehungsweise Sortieren wird unterstützt. 

Alle anderen Architekturen erfüllen mindestens ein Kriterium nicht und werden deswegen nicht verwendet.

Für CROSSLOAD wird sich für die Architektur \textit{IndexedDB} entschieden. Die Verwendung von \textit{Cache API und IndexedDB} wäre auch möglich, verkompliziert die Anwendung jedoch unnötig. Das Verwenden von zwei \acp{API} macht den Code für andere Entwickler komplizierter aber bringt keinen Vorteil gegenüber der Verwendung von IndexedDB alleine.

\section{Herunterladen der Audiodaten}
Favorisierte Inhalte sollen für den Offline-Gebrauch heruntergeladen werden. Dafür gibt es verschiedene Möglichkeiten. In diesem Kapitel werden verschiedene Funktionen vorgestellt und miteinander verglichen. Zum Vergleichen dienen die Kriterien: Mit welchen Browsern ist die \ac{API} kompatibel, wie ist die Lebensdauer des Downloads (nur wenn der Tab geöffnet ist, Lebensdauer eines Service Workers, ...) und wie leicht ist die Technologie in die bestehende Anwendung integrierbar.

\subsection{Fetch API}
Die Fetch \ac{API} bietet die Möglichkeit, XMLHttpRequests zu verschicken, also Ressourcen von einem Server anzufragen \autocite{Rojas2020}. Verwendet wird die \ac{API} mit Promises, ist also asynchron und bietet eine einfachere Fehlerbehandlung \autocite{Rojas2020} \autocite{mdn-fetch}. 

Außerdem unterstützt jeder relevante Browser diese Funktion \autocite{mdn-fetch}. Die Fetch \ac{API} kann sowohl im \ac{DOM} als auch in Service Workern verwendet werden \autocite{mdn-fetch}. Das heißt, dass in einem Angular Service oder in Komponenten der Anwendung eine Netzwerkanfrage mit der Fetch \ac{API} versendet werden kann.

Wenn die Fetch \ac{API} in einem Service Worker verwendet wird, ergeben sich einige Vorteile. Ein Service Worker kann begrenzt im Hintergrund weiterlaufen, auch wenn die Webanwendung geschlossen ist. Jedoch kann ein Service Worker vom Browser zu jeder Zeit beendet werden, selbst wenn ein Event noch nicht beendet wurde \autocite{service-worker-spec}. Die Ausführungsdauer für ein Event sollte sehr kurz sein und ist nicht für langanhaltende Downloads gedacht \autocite{service-worker-spec}.

Bis jetzt bietet Angular keinen Weg, einen eigenen Service Worker zu schreiben, der gut in die Anwendung integrierbar ist, also zum Beispiel Funktionen eines Services aufrufen kann. 

\subsection{HttpClient}
In Angular können Netzwerkanfragen mit dem HttpClient verschickt werden \autocite{httpClient}. Bei einem Request können die URL, Header, Parameter und vieles mehr angepasst werden \autocite{httpClient}. Die Antwort eines Request wird als Observable zurück gegeben. Der HttpClient bietet ebenso die Option, zwischendurch den Fortschritt des Downloads abzufragen \autocite{httpClient}. Dadurch, dass der HttpClient ein Teil von Angular ist, funktioniert er in allen relevanten Browsern und ist sehr leicht in die bestehende Anwendung integrierbar. 

Der HttpClient kann in allen Angular Komponenten und Services per Dependecie Injection genutzt werden \autocite{httpClient}. In einem Service Worker können keine angularspezifischen Teile verwendet werden, somit ist auch der HttpClient in Service Workern nicht nutzbar \autocite{service-worker-angular}.

\subsection{Background Sync}
\bild{diagramme/background-sync}{14cm}{Ablauf für Background Sync}

Mit Background Sync können Netzwerkanfragen abgeschickt werden, selbst wenn keine Internetverbindung besteht. Die Anfrage wird gespeichert und sobald eine Internetverbindung besteht abgeschickt \autocite{wicg-background-sync} \autocite{Rojas2020}. 

\autoref{diagramme/background-sync} zeigt den Ablauf, wenn ein Sync-Event gesendet wird. Wenn bereits eine Netzwerkverbindung besteht, wird das Event sofort ausgeführt. Ansonsten wird auf eine Netzwerkverbindung gewartet. Das Ausführen des Events kann zum Beispiel durch einen Netzwerkfehler scheitern. Wenn nach mehrmaligen Versuchen das Event nicht geklappt hat, wird ein \textit{Last-Chance-Event} gefeuert \autocite{Rojas2020}.

Diese Funktionalität ist Teil von Service Workern und funktioniert, solange ein Service Worker läuft \autocite{wicg-background-sync}. Auch das Event für Background Sync hat nur eine begrenzte Lebensdauer und kann jederzeit vom Browser beendet werden \autocite{service-worker-spec}. Auch diese Funktionalität von Service Workern ist noch nicht in Angular integriert. Bei der Verwendung dieser Funktionalität können also keine Angular-Komponenten, beziehungsweise Funktionen zum Einsatz kommen.

Im Moment wird diese Funktion nur von Chrome unterstützt \autocite{caniuse-background-sync}, in Firefox ist sie aber bereits in Entwicklung \autocite{status-mozilla-background-sync}.

\subsection{Background Fetch}
Diese \ac{API} bietet die Möglichkeit, große Dateien hoch- oder herunterzuladen und zeigt dem Nutzer dabei Informationen über den Fortschritt \autocite{google-background-fetch}. Der Unterschied zu Background Sync liegt darin, dass der Download, beziehungsweise Upload auch funktioniert, wenn der Browser geschlossen wird und kein Service Worker aktiv ist \autocite{google-background-fetch}. \autoref{screenshots/background-fetch-benachrichtigung-android} zeigt die Fortschrittsanzeige auf einem Android Smartphone. Viele nützliche Funktionen, wie beispielsweise pausieren und fortsetzen des Downloads bei Verbindungsunterbrechungen werden ebenso von Background Fetch übernommen \autocite{google-background-fetch}. 

\bild{screenshots/background-fetch-benachrichtigung-android}{8cm}{Benachrichtigung auf Android bei Benutzung der Background Fetch \ac{API}}

Diese \ac{API} ist im Moment nur in Chrome verfügbar \autocite{google-background-fetch}.

Auch Background Fetch ist noch nicht in Angular integriert und Bibliotheken zur Verwendung konnten auch nicht gefunden werden. Somit ist die Integration in die bestehende Anwendung kompliziert.

\subsection{Entscheidung}

\begin{table}[h]
  \renewcommand{\arraystretch}{1.2}
  \centering
  \sffamily
  \begin{footnotesize}
    \begin{tabularx}{0.9\textwidth}{l L L L}
      \toprule
      \textbf{Technologie} & \textbf{Browserkompatibilität} & \textbf{Lebensdauer} & \textbf{Integrierbarkeit} \\
      \midrule
      \emph{Fetch \ac{API}} & alle & Lebensdauer eines Service Workers & in Service Workern schlecht integrierbar, ansonsten gut \\
      \emph{HttpClient} & alle & Solange die Anwendung geöffnet ist & sehr gut \\
      \emph{Background Sync} & Chrome & Lebensdauer eines Service Workers & schlecht \\
      \emph{Background Fetch} & Chrome & Bis der Download abgeschlossen ist & schlecht \\
      \bottomrule
    \end{tabularx}
  \end{footnotesize}
  \rmfamily
  \caption{Vergleich der Technologien zum Herunterladen von Dateien}
  \label{Kap4:DownloadTechnologien}
\end{table}

Sowohl der HttpClient als auch die Fetch \ac{API} laufen in allen relevanten Browsern. Die Fetch \ac{API} und Background Sync erlauben einen längeren Download im Hintergrund solange der Service Worker noch aktiv ist, selbst wenn der Tab nicht geöffnet ist. Der HttpClient ist dagegen leichter zu integrieren.

Als Technologie wird für CROSSLOAD der HttpClient ausgewählt. Dieser ist am einfachsten zu integrieren und läuft in allen Browsern. Die längere Lebensdauer der Fetch \ac{API} ist nur minimal, weil ein Service Worker keine langen Berechnungen, beziehungsweise Downloads durchführen soll.

Als Ergänzung kann später Background Fetch implementiert werden, weil damit auch Inhalte heruntergeladen werden können, wenn die Website geschlossen ist. Dies ist jedoch nicht Teil dieser Thesis.

\section{Verbindungsstaus auslesen}
In Anforderung 2 und 4 in \autoref{Kap3:Anforderungen} wurden Anforderungen definiert, die ein Auslesen des Verbindungsstatus erfordern. Dazu gehört die Frage, ob überhaupt eine Internetverbindung besteht und auch, ob der Nutzer über Mobile Daten oder WLAN verbunden ist. Außerdem soll auf eine Änderung dieser Werte möglichst schnell reagiert werden können.

In diesem Kapitel wird untersucht, ob Funktionen zum Auslesen dieser Daten existieren und welche Browser diese unterstützen. 

\subsection{Online / Offline}
Mit \emph{navigator.onLine} gibt es in allen relevanten Browsern \autocite{caniuse-online} eine Funktion, um zu überprüfen, ob eine Internetverbindung besteht oder nicht \autocite{Sheppard2017} \autocite{mdn-online}. Diese \ac{API} arbeitet jedoch nicht ganz genau. Wenn eine Verbindung zu einem Netzwerk besteht, in diesem Netzwerk aber kein Internetzugriff möglich ist, gibt \emph{navigator.onLine} in manchen Fällen true zurück und signalisiert eine Internetverbindung \autocite{Sheppard2017}. Es kann aber nicht vorkommen, dass der Nutzer offline ist und die \ac{API} true zurück liefert. Dies ist eine zu vernachlässigende Ungenauigkeit, weil dieser Fall nur sehr selten auftritt. 

Wenn eine zuverlässigere Methode benötigt wird, kann ein einfacher Netzwerk-Request abgeschickt werden. Wenn ein Ergebnis zurück kommt, existiert eine Verbindung, wenn ein Fehler geworfen wird, existiert keine Verbindung. 

Des Weiteren gibt es auch eine Möglichkeit, einen Verbindungsstatuswechsel mitzubekommen. Es gibt zwei Events, auf die man reagieren kann: \emph{online} wenn der Browser eine Internetverbindung herstellen konnte und \emph{offline} wenn eine bestehende Internetverbindung abbricht \autocite{mdn-online} \autocite{Freeman2020}. Auch hier treffen die Limitierungen von \emph{navigator.onLine} zu.

\subsection{Verbindungsart}
Zusätzlich ist es erforderlich, die aktuelle Verbindungsart des Gerätes auslesen zu können. Das könnte zum Beispiel WLAN oder Mobilfunk sein. In Anforderung 2 aus \autoref{Kap3:Anforderungen} wird mit einem Download so lange gewartet, bis eine WLAN-Verbindung besteht oder der Nutzer zugestimmt hat. Mit \emph{navigator.connection.type} gibt es eine \ac{API} dies zu erreichen \autocite{network-type-mdn}. Der Wert \textit{cellular} wird zurückgegeben, wenn eine Mobilfunkverbindung besteht \autocite{network-type-mdn}. Es ist auch möglich, auf ein Event zu hören, sobald sich dieser Status ändert \autocite{network-information-mdn}. Diese \ac{API} ist aber noch nicht in allen relevanten Browsern verfügbar. Nur Nutzer die Chrome verwenden, können von dieser \ac{API} profitieren \autocite{network-type-mdn}.

Auch wenn nur Nutzer von Chrome diese \ac{API} nutzen können wird das Herunterladen anhand der Verbindungsart in CROSSLOAD umgesetzt werden. Es ist wahrscheinlich, dass andere Browser die fehlende \ac{API} in Zukunft auch unterstützten werden.
