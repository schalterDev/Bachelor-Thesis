\chapter{Konzeption}
\label{Kap4}

\section{Lokale Datenspeicherung}
Moderne Webbrowser stellen viele verschiedene Lösungen zum lokalen Speichern von Daten zu Verfügung. In diesem Kapitel werden einige davon mit ihren Stärken und Schwächen vorgestellt. Zuletzt wird eine passende Lösung für CROSSLOAD ausgewählt und begründet.

\subsection{Web Storage}
Der Web Storage besitzt zwei unterschiedliche Arten Daten zu speichern: LocalStorage und SessionStorage \autocite{Hajian2019} \autocite{mdn-web-storage}. Die beiden Speicher unterscheiden sich nur in der Dauer der Speicherung der Daten. Die Daten vom SessionStorage werden gelöscht sobald die Sitzung auf der Website vorbei ist, das heißt der Browser oder der Tab geschlossen wird \autocite{Hajian2019} \autocite{mdn-web-storage}. Der LocalStorage bleibt über mehrere Sitzungen erhalten und wird nicht automatisch gelöscht \autocite{Hajian2019} \autocite{mdn-web-storage}. 

Die Web Storage \ac{API} bietet die Möglichkeit Schlüssel / Wert Paare im Browser abzulegen \autocite{mdn-web-storage}. Alle relevanten Browser implementieren diese \ac{API} \autocite{mdn-web-storage}, es gibt aber auch einige Einschränkungen:

\begin{itemize}
	\item Zugriffe sind nur synchron möglich \autocite{Hajian2019}
	\item Als Schlüssel und Werte können jeweiles nur Strings gespeichert werden \autocite{Hajian2019}
	\item Die \ac{API} ist nicht von Web Workern aufrufbar \autocite{Hajian2019}. Service Worker haben zum Beispiel keinen Zugriff darauf
	\item Es können maximal 5 \ac{MB} an Daten gespeichert werden \autocite{mdn-web-storage}
\end{itemize}

Aufgrund der synchronität und des geringen Datenvolumes das gespeichert werden darf, eignet sich die Web Storage \ac{API} nur zum Speichern von wenig Daten. Größere Daten und insbesondere binäre Dateien, die nicht sehr gut in eine textuelle Form übertragen werden können sollten nicht abgespeichert werden. 

\subsection{File System API}
Die File System \ac{API} und FileWrite \ac{API} bieten dem Browser die Möglichkeit Dateien in ein virtuelles Dateisystem abzulegen und von dort wieder zu laden \autocite{Hajian2019} \autocite{storage-for-the-web}. Dieses virtuelle Dateisystem unterstützt viele gängige Funktionen wie Ordner und Dateimanipulationen, wie man sie von herkömmlichen Dateisystem gewohnt ist \autocite{mdn-file-system}. MDN nennt diese \ac{API} File and Directory Entries \ac{API} \autocite{mdn-file-system}. 

Der Vorteil dieser \ac{API} ist der Umgang mit \acp{blob}, wie zum Beispiel Audiodateien. Diese können leicht gespeichert, geladen und manipuliert werden \autocite{mdn-file-system}. Außerdem gibt es eine \ac{API} für synchrone und asynchrone Zugriffe und ist ebenso in Web Workern verfügbar \autocite{Hajian2019}. 

Es gibt keinen offiziellen Standard \autocite{mdn-file-system} \autocite{caniuse-filesystem}. Deswegen unterstützen noch nicht sehr viele Browser diese \ac{API}, bisher ist in Chrome und in allen Chromium basierten Browser diese Funktion verfügbar \autocite{caniuse-filesystem}.

\subsection{Web SQL}
Web SQL ist eine \ac{API}, die es erlaubt Daten in einer Datenbank zu speichern und diese Daten mit einer SQL ähnlichen Sprache zu durchsuchen \autocite{w3-web-sql}. Die Anfragen sind asynchron funktionieren aber nicht in einem Web Worker \autocite{Hajian2019}.

Diese \ac{API} wurde nie von allen Browsern implementiert und ist mitlerweile deprecated, soll also nicht mehr verwendet werden \autocite{Hajian2019}.

Aufgrund der Tatsache, dass Web SQL deprecated ist, wird diese Technologie auch keinen Einsatz bei CROSSLOAD finden. Als Datenbank in Browsern hat sich IndexedDB etabliert.

\subsection{Indexed DB}
Indexed DB ist eine key-value NoSQL objekt-orientierte Datenbank zum Speichern von großen und vielen Dateien \autocite{Hajian2019}. Konkret heißt das, dass Objekte mit einem Schlüssel in die Datenbank abgelegt werden können und diese über den angegebenen Schlüssel wieder auffindbar sind. Dabei werden viele unterschiedliche Datentypen (boolean, number, string, date, object, array, regexp, undefined, null, \ac{blob}) \autocite{mdn-indexeddb}, Transaktionen und Indexe zum schnelleren durchsuchen unterstützt \autocite{Sheppard2017}. 

Einschränkungen in der Verwendung der \ac{API} gibt es nicht, da IndexedDB auch in Web Worker zur Verfügung steht und asynchron verwendet werden kann \autocite{Hajian2019} \autocite{mdn-indexeddb}. Alle relevanten Browser unterstützen Indexed DB \autocite{mdn-indexeddb-api}. Die maximale Speicherkapazität der Datenbank hängt vom Browser ab und wird in Kapitel ???näher behandelt.

Anfragen an die Datenbank können direkt eine Sortierung oder Suchkriterien enthalten \autocite{mdn-indexeddb}. Dabei ist zu beachten, dass die Textsuche limitiert ist und keine Anfragen zum Finden eines einzelnen Wortes in einem ganzen Text unterstützt \autocite{mdn-indexeddb}. 

Die Verwendung von Indexed DB ist komplex, deswegen existieren viele Bibliotheken, die eine Verwendung der \ac{API} erleichtern wollen, wie zum Beispiel LocalForage oder Dexie.js \autocite{Hajian2019} \autocite{mdn-indexeddb}.

Indexed DB ist ein sehr flexibler Speicher für den Client, weil in einer Anfrage direkt gesucht und sortiert werden kann. Außerdem können simple Datentypen, Objekte und sogar \acp{blob} (binäre Dateien) gespeichert werden. Ein Nachteil ist die Komplexität in der Verwendung der \ac{API}.

\subsection{Cache API}
- Service Worker: Laufen im Hintergrund (im eigenen Thread) ohne Zugriff aufs DOM \autocite{Sheppard2017}

- Service Worker ist ein Mittler zwischen APP und Internet. Es übernimmt Aufgaben wie Caching, Syncing, Benachrichtigungen \autocite{Sheppard2017}

Service Worker:
- "Arbeiter" in JavaScript die im Hintergrund laufen
- Ermöglicht es Netzwerk-Request zu unterbrechen (Proxy)
- Benachrichtigungen zu senden
- Cache Verwalten und vieles mehr
- Laufen im eigenen Thread ohne DOM Zugriff, HTTPS-Only
- Website sollte nicht auf Service Worker angewiesen sein
\autocite{Hajian2019}

Angular bietet einen einfachen Weg Service Workers zu verwenden:
- ng add @angular/pwa
\autocite{Hajian2019}

Cache Strategien:
- Cache only: Es wird nur der Inhalt des Caches geliefert. Wenn nicht verfügbar wird nichts geliefert
- Netzwerk only: Es wird nur die Antwort des Requests gesendet
- Cache-First: Cache und wenn nicht verfügbar -> Netzwerk
- Netzwerk-First: Netzwerk und wenn nicht verfügbar -> Cache
- Cache und Netzwerk (stale-while-revalidate): Nimm erst den Wert vom Cache und sobald das Netzwerk geantwortet hat, die Antwort vom Netzwerk. Gut für sich häufig ändernde Daten (Facebook)
- Generic-Fallback: Zum Beispiel für ein Bild ein Standardbild liefern, wenn Netzwerk und Cache keins haben
\autocite{Hajian2019} \autocite{Rojas2020}

Bibliothek für Service Worker:
- Workbox
\autocite{Rojas2020}

Cache storage != browser cache \autocite{Rojas2020}

\subsection{Überblick und Entscheidung}
In Tabelle~\ref{Kap4:Datenspeicherung} werden die vorgestellten Möglichkeiten zum Speichern von Daten nach folgenden Kriterien verglichen: maximale Datenmenge, mögliche Datentypen, synchron / asynchron, Browsersupport, in Web Worker aufrufbar. Bei fehlenden Feldern in Web SQL wurde nicht weiter recherchiert, weil diese Technologie veraltet ist und nicht mehr verwendet werden soll.

\begin{sidewaystable}[h]
  \caption{Vergleich der APIs zur lokalen Datenspeicherung}
  \label{Kap4:Datenspeicherung}
  \renewcommand{\arraystretch}{1.2}
  \centering
  \sffamily
  \begin{footnotesize}
    \begin{tabularx}{1.0\textwidth}{l l l l l l l}
      \toprule
      \textbf{Technologie} & \textbf{Datenmenge} & \textbf{Datentypen} & \textbf{(a)synchron} & \textbf{Browsersupport} & \textbf{Web Worker} & \textbf{Anmerkung} \\
      \midrule
      \emph{Web Storage \ac{API}} & max. 5 \ac{MB} & String & nur synchron & alle & nein & \\
      \emph{File System \ac{API}} & quota  & \acp{blob} und weitere & synchron und asynchron & nur Chrome & Ja &  \\
      \emph{Web SQL} & - & . & synchron und asynchron & teilweise & nein & veraltet \\
      \emph{Indexed DB} & quota? & einfache Datentypen, Objekte, \acp{blob} & asynchron & alle & Ja & Inhalt durchsuch- und filterbar \\
      \emph{Cache \ac{API}} &  &  &  &  &  & \\
      \bottomrule
    \end{tabularx}
  \end{footnotesize}
  \rmfamily
\end{sidewaystable}

\section{Herunterladen der Audiodaten}

\subsection{Fetch API}

Einfachere XMLHttpRequests, Promise basierend
\autocite{Rojas2020}

\subsection{Background Fetch}

\subsection{Background Sync}

- Speichert den API Call bis eine stabile Internetverbindung besteht und schickt ihn dann raus. Sogar wenn die APP nicht aktiv ist oder läuft \autocite{Sheppard2017}

- Versendet Request sobald eine Internetverbindung besteht, auch wenn die APP nicht offen ist \autocite{Rojas2020}

- Man kann diese API mit IndexedDB kombinieren. Daten können in der Datenbank zwischengespeichert werden, bis eine Verbindung besteht \autocite{Rojas2020}

\section{Verbindungsstaus auslesen}

- navigator.onLine: ist nicht 100\% genau. Wird manchmal sagen, dass das Gerät online ist wenn es nur mit einem Netzwerk verbunden ist \autocite{Sheppard2017}


