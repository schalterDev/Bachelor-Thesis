\chapter{Grundlagen}
\label{Kap2}

\section{Typescript}

\section{Angular}

\section{Progressive Web App}
- Liste von Strategien, Techniken und APIs um dem Nutzer eine Erfahrung ähnlich zu nativen APPs zu bieten \autocite{Sheppard2017}

- PWAs sind: Schnell, Zuverlässig (auch ohne Internet und auf alten Geräten), Engaging? \autocite{Sheppard2017} \autocite{Hajian2019}

- Features von PWAs: Offline support (zumindest für die Hauptseite), Schnell (auch für mobiles Internet), Home screen icon und splash screen, Benachrichtigungen \autocite{Sheppard2017}

Features von PWAs:
- Sehr schnell geladen und Interaktion möglich
- offlien fähig
- Responsive: mobile-first (weil es schlechtere Hardware hat, offline-first)
- Benachrichtigungen
- Native-like Features: hardware apis (kamera, bluetooth, ...)
- Sicher (HTTPS)
- Installierbar
- Fortschrittlich (progressive)
\autocite{Hajian2019}

PWAs haben ein APP Manifest mit
- name, icon, usw.
\autocite{Hajian2019}

PWAs user experience ist zwischen web interaktion und einer mobilen app
\autocite{Rojas2020}

Vorteile einer PWA:
- Webtechnologien sind weit verbreitet und in allen Betriebssystemen verfügbar
- Entwicklung in Web Technologien ist schnell und es gibt viele kostenlose Tools
- Kann leicht über eine URL ausgeliefert werden
- Leichtes Deployement: kein Review durch eine Firma
- kein App Store, der am Anfang für Entwickler stressig und umständlich ist
\autocite{Rojas2020}

PWA besteht aus drei Teilen:
- Manifest Datei: Informationen über PWA
- App Icon: Wird angezeigt wenn die APP installiert wird
- Service Workers: 
\autocite{Rojas2020}

\section{CROSSLOAD}
