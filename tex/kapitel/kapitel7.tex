\chapter{Zusammenfassung und Ausblick}
\label{Kap7}
Dieses Kapitel gibt eine Zusammenfassung der ganzen Arbeit. Zuletzt werden auch noch einige Punkte genannt, die in Zukunft angedacht werden könnten, um CROSSLOAD weiter zu verbessern.

\section{Zusammenfassung}
Das Ziel dieser Arbeit war es, das bereits vorhandene Webportal CROSSLOAD um eine Offline-Funktionalität zu erweitern. Außerdem wurde dabei untersucht, welche Webtechnologien dafür zur Verfügung stehen und ob diese ausreichen, um alle Anforderungen zu erfüllen. Dem Nutzer soll es möglich sein, Inhalte des Portals im Browser zu speichern und diese Inhalte auch nutzen zu können, wenn keine oder eine schlechte Internetverbindung besteht. Moderne Browser bieten die Möglichkeit, Daten in einer Datenbank der IndexedDB zu speichern, die auch ein Filtern und Durchsuchen erlaubt. In dieser Datenbank werden Inhalte und deren Metadaten gespeichert. Der Download der Inhalte findet in einem Angular Service statt. Es gibt andere Technologien, wie Background Fetch, die dem Nutzer eine bessere Erfahrung bieten würden, diese sind aber noch nicht für alle Browser verfügbar. 

Zum Bereitstellen der Inhalte und des Downloadfortschrittes wird immer auf das Publish-Subscribe Pattern gesetzt. Dadurch ist es sehr flexibel möglich, Daten an mehrere Komponenten der grafischen Bedienoberfläche weiterzugeben und diese jederzeit aktuell zu halten. Die grafische Oberfläche für diese Funktionen wurde noch nicht fertig entwickelt, sondern nur die Machbarkeit des Vorhabens gezeigt. Entwürfe für das endgültige Design sind für spätere Entwicklungen vorhanden. 

Zusammenfassend kann gesagt werden, dass Webtechnologien und \acp{PWA} sehr gut geeignet sind, um dem Nutzer auch eine Offline-Nutzung der Anwendung zu ermöglichen. Fast alle Anforderungen konnten ohne Kompromisse umgesetzt werden. Nur eine Anforderung ist noch nicht in allen Browsern umsetzbar. Die Entwicklung wird dadurch erschwert, dass nicht jeder Browser alle Funktionen implementiert. Dazu gehört zum Beispiel die Möglichkeit, die Verbindungsart wie WLAN oder mobile Daten auszulesen, um den Download von Inhalten zu steuern. Außerdem bieten \acp{PWA} noch nicht ausreichend die Möglichkeit, Aktionen im Hintergrund auszuführen, auch wenn der Browser zwischenzeitlich geschlossen wird. Ein Download von Inhalten im Hintergrund, während der Browser also geschlossen, beziehungsweise minimiert ist, ist deswegen nicht auf allen Geräten möglich. Diese Funktionalität kann einigen Nutzern zur Verfügung gestellt werden, eine Alternative für nicht unterstützte Geräte muss aber trotzdem entwickelt werden. Wie die neuen Funktionalitäten vom Nutzer angenommen werden, ist noch zu beobachten und aufgrund dieser Erkenntnisse ist es wichtig, das Portal immer wieder anzupassen.

\section{Ausblick}
Nach dem Abschluss dieser Arbeit gibt es noch viele Möglichkeiten, CROSSLOAD zu verbessern. Dazu gehört zuerst die Implementierung der erarbeiteten Designs. Die Designs müssen auch regelmäßig überdacht und evaluiert werden und dementsprechend das Portal angepasst werden. Außerdem ist eine Sortierung, Filterung und das Durchsuchen der Offline-Inhalte noch nicht implementiert. Dies ist vor allem für Nutzer mit sehr vielen Inhalten nützlich. 

Beim Beobachten einiger Nutzer ist aufgefallen, dass Inhalte schnell heruntergeladen werden, aber dann vergessen werden, zu löschen. Dies führt dann dazu, dass der Speicher knapp wird und eventuell sogar dazu, dass keine neuen Inhalte mehr heruntergeladen werden können. Dafür könnte eine Strategie entwickelt werden, wann Inhalte automatisch gelöscht werden. Alternativ kann dem Nutzer selbst die Möglichkeit gegeben werden einzustellen, wann Inhalte automatisch gelöscht werden sollen. Das könnte in verschiedene Kriterien wie \textit{wurde die Predigt schon angehört} oder \textit{wann wurde der Inhalt heruntergeladen} unterteilt werden. Anhand dieser Kriterien kann eine Bewertung erstellt werden, wie relevant der Inhalt noch für den Nutzer ist. Die Inhalte mit der geringsten Bewertung werden gelöscht, sobald Speicher benötigt wird.

Zuletzt sollte eine Synchronisation der heruntergeladenen Inhalte über mehrere Geräte entwickelt werden. Ein Nutzer möchte seine Inhalte vielleicht auf dem Laptop auswählen und dann unterwegs auf seinem Smartphone anhören. Der Nutzer legt sich dafür einen Account bei CROSSLOAD an oder meldet sich über einen anderen Dienst wie Google oder Facebook bei CROSSLOAD an. Die vorgemerkten Inhalte werden dann auf einem Server von CROSSLOAD für jeden Nutzer gespeichert. Bei jedem Start des Portals, auf dem Smartphone oder Laptop, wird die Liste der vorgemerkten Inhalte geladen und gegebenenfalls neue Inhalte heruntergeladen. 
