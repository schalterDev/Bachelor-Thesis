\chapter{Einleitung}
\label{Kap1}

In diesem Kapitel wird zuerst die Motivation und Zielsetzung der vorliegenden Thesis erläutert, damit der Leser sich einen ersten Eindruck verschaffen kann. Anschließend gibt es einen Überblick über alle nachfolgenden Kapitel.


\section{Motivation}
Das Webportal CROSSLOAD bietet viele christliche Medien zum Download an. Zur Nutzergruppe gehören Leute, die es gewohnt sind mit dem Computer oder Smartphone zu arbeiten, wie zum Beispiel Studenten. Aber auch ältere Menschen, die erst seit ein paar Monaten ein Smartphone besitzen und somit wenig Erfahrung im Umgang mit technischen Geräten haben, gehören zur Zielgruppe. Das Portal wird sowohl von daheim im eigenen WLAN als auch unterwegs mit mobilen Daten genutzt.

Durch die mancherorts schlechte Mobilfunkabdeckung oder das beschränkte Datenvolumen möglicher Nutzer, könnte die Nutzung des Portals an vielen Stellen nicht möglich sein. Bei dem Streamen von Inhalten im Auto oder im Zug könnte es wegen Verbindungsunterbrechungen zu Verzögerungen kommen. Wenn man die Medien unterwegs abrufen möchte, müsste man sie sich vorher herunterladen und anschließend auf dem Gerät suchen. Für einige Nutzer ist das eine zu große Hürde und muss deswegen vereinfacht werden. 

\section{Zielsetzung}
Ziel dieser Arbeit ist es für CROSSLOAD ein Konzept zur Offline Nutzung zu entwickeln und anschließend zu implementieren. Inhalte die Offline genutzt werden sollen sind Predigten im Audio-Format, die bis zu 90 Minuten lang sind. Allen Nutzern soll eine komfortable Nutzung des Webportals angeboten werden, sowohl auf Desktop-Computern, Tablets oder Smartphones. 

Dafür werden verschiedene Funktionen und \acp{API} von Webbrowsern untersucht und verglichen. Die passendsten Funktionen werden ausgewählt und prototypisch in das vorhandene Portal eingebunden.

Schließlich stellt sich noch die Frage, ob die Funktionen und APIs der Webbrowser ausreichen, um alle Anforderungen zu erfüllen. Ist eine \ac{PWA} in der Lage für CROSSLOAD eine native App zu ersetzen?

\section{Aufbau}
Zuerst werden die Anforderungen an das Webportal im Blick auf die Offline Nutzung herausgearbeitet. Danach folgt die Konzeption zur Datenspeicherung und zum Herunterladen der Daten. Anschließend wird auf die Implementierung ausgewählter Funktionen eingegangen. In Kapitel 6 folgt die Evaluation und kritische Beurteilung. Zuletzt werden noch einige Funktionen beleuchtet, die in Zukunft umgesetzt werden könnten, um eine noch bessere Nutzerfreundlichkeit zu bieten.
